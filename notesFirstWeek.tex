% Options for packages loaded elsewhere
\PassOptionsToPackage{unicode}{hyperref}
\PassOptionsToPackage{hyphens}{url}
%
\documentclass[
]{article}
\usepackage{amsmath,amssymb}
\usepackage{lmodern}
\usepackage{ifxetex,ifluatex}
\ifnum 0\ifxetex 1\fi\ifluatex 1\fi=0 % if pdftex
  \usepackage[T1]{fontenc}
  \usepackage[utf8]{inputenc}
  \usepackage{textcomp} % provide euro and other symbols
\else % if luatex or xetex
  \usepackage{unicode-math}
  \defaultfontfeatures{Scale=MatchLowercase}
  \defaultfontfeatures[\rmfamily]{Ligatures=TeX,Scale=1}
\fi
% Use upquote if available, for straight quotes in verbatim environments
\IfFileExists{upquote.sty}{\usepackage{upquote}}{}
\IfFileExists{microtype.sty}{% use microtype if available
  \usepackage[]{microtype}
  \UseMicrotypeSet[protrusion]{basicmath} % disable protrusion for tt fonts
}{}
\makeatletter
\@ifundefined{KOMAClassName}{% if non-KOMA class
  \IfFileExists{parskip.sty}{%
    \usepackage{parskip}
  }{% else
    \setlength{\parindent}{0pt}
    \setlength{\parskip}{6pt plus 2pt minus 1pt}}
}{% if KOMA class
  \KOMAoptions{parskip=half}}
\makeatother
\usepackage{xcolor}
\IfFileExists{xurl.sty}{\usepackage{xurl}}{} % add URL line breaks if available
\IfFileExists{bookmark.sty}{\usepackage{bookmark}}{\usepackage{hyperref}}
\hypersetup{
  pdftitle={notes},
  hidelinks,
  pdfcreator={LaTeX via pandoc}}
\urlstyle{same} % disable monospaced font for URLs
\usepackage[margin=1in]{geometry}
\usepackage{color}
\usepackage{fancyvrb}
\newcommand{\VerbBar}{|}
\newcommand{\VERB}{\Verb[commandchars=\\\{\}]}
\DefineVerbatimEnvironment{Highlighting}{Verbatim}{commandchars=\\\{\}}
% Add ',fontsize=\small' for more characters per line
\usepackage{framed}
\definecolor{shadecolor}{RGB}{248,248,248}
\newenvironment{Shaded}{\begin{snugshade}}{\end{snugshade}}
\newcommand{\AlertTok}[1]{\textcolor[rgb]{0.94,0.16,0.16}{#1}}
\newcommand{\AnnotationTok}[1]{\textcolor[rgb]{0.56,0.35,0.01}{\textbf{\textit{#1}}}}
\newcommand{\AttributeTok}[1]{\textcolor[rgb]{0.77,0.63,0.00}{#1}}
\newcommand{\BaseNTok}[1]{\textcolor[rgb]{0.00,0.00,0.81}{#1}}
\newcommand{\BuiltInTok}[1]{#1}
\newcommand{\CharTok}[1]{\textcolor[rgb]{0.31,0.60,0.02}{#1}}
\newcommand{\CommentTok}[1]{\textcolor[rgb]{0.56,0.35,0.01}{\textit{#1}}}
\newcommand{\CommentVarTok}[1]{\textcolor[rgb]{0.56,0.35,0.01}{\textbf{\textit{#1}}}}
\newcommand{\ConstantTok}[1]{\textcolor[rgb]{0.00,0.00,0.00}{#1}}
\newcommand{\ControlFlowTok}[1]{\textcolor[rgb]{0.13,0.29,0.53}{\textbf{#1}}}
\newcommand{\DataTypeTok}[1]{\textcolor[rgb]{0.13,0.29,0.53}{#1}}
\newcommand{\DecValTok}[1]{\textcolor[rgb]{0.00,0.00,0.81}{#1}}
\newcommand{\DocumentationTok}[1]{\textcolor[rgb]{0.56,0.35,0.01}{\textbf{\textit{#1}}}}
\newcommand{\ErrorTok}[1]{\textcolor[rgb]{0.64,0.00,0.00}{\textbf{#1}}}
\newcommand{\ExtensionTok}[1]{#1}
\newcommand{\FloatTok}[1]{\textcolor[rgb]{0.00,0.00,0.81}{#1}}
\newcommand{\FunctionTok}[1]{\textcolor[rgb]{0.00,0.00,0.00}{#1}}
\newcommand{\ImportTok}[1]{#1}
\newcommand{\InformationTok}[1]{\textcolor[rgb]{0.56,0.35,0.01}{\textbf{\textit{#1}}}}
\newcommand{\KeywordTok}[1]{\textcolor[rgb]{0.13,0.29,0.53}{\textbf{#1}}}
\newcommand{\NormalTok}[1]{#1}
\newcommand{\OperatorTok}[1]{\textcolor[rgb]{0.81,0.36,0.00}{\textbf{#1}}}
\newcommand{\OtherTok}[1]{\textcolor[rgb]{0.56,0.35,0.01}{#1}}
\newcommand{\PreprocessorTok}[1]{\textcolor[rgb]{0.56,0.35,0.01}{\textit{#1}}}
\newcommand{\RegionMarkerTok}[1]{#1}
\newcommand{\SpecialCharTok}[1]{\textcolor[rgb]{0.00,0.00,0.00}{#1}}
\newcommand{\SpecialStringTok}[1]{\textcolor[rgb]{0.31,0.60,0.02}{#1}}
\newcommand{\StringTok}[1]{\textcolor[rgb]{0.31,0.60,0.02}{#1}}
\newcommand{\VariableTok}[1]{\textcolor[rgb]{0.00,0.00,0.00}{#1}}
\newcommand{\VerbatimStringTok}[1]{\textcolor[rgb]{0.31,0.60,0.02}{#1}}
\newcommand{\WarningTok}[1]{\textcolor[rgb]{0.56,0.35,0.01}{\textbf{\textit{#1}}}}
\usepackage{graphicx}
\makeatletter
\def\maxwidth{\ifdim\Gin@nat@width>\linewidth\linewidth\else\Gin@nat@width\fi}
\def\maxheight{\ifdim\Gin@nat@height>\textheight\textheight\else\Gin@nat@height\fi}
\makeatother
% Scale images if necessary, so that they will not overflow the page
% margins by default, and it is still possible to overwrite the defaults
% using explicit options in \includegraphics[width, height, ...]{}
\setkeys{Gin}{width=\maxwidth,height=\maxheight,keepaspectratio}
% Set default figure placement to htbp
\makeatletter
\def\fps@figure{htbp}
\makeatother
\setlength{\emergencystretch}{3em} % prevent overfull lines
\providecommand{\tightlist}{%
  \setlength{\itemsep}{0pt}\setlength{\parskip}{0pt}}
\setcounter{secnumdepth}{-\maxdimen} % remove section numbering
\ifluatex
  \usepackage{selnolig}  % disable illegal ligatures
\fi

\title{notes}
\author{}
\date{\vspace{-2.5em}}

\begin{document}
\maketitle

\hypertarget{intro.}{%
\subsection{Intro.}\label{intro.}}

Using the mpg data set, we can get a start to fucking up daddy fung.
Make sure u get all the releveant libraries coz ur gonna need to be
using them.

\hypertarget{ytb}{%
\subsection{ytb}\label{ytb}}

to load up the data, just write the name of the data set in the console,
it'll give a fat tibble table. if u dunno anything, just put a ? mark
inf front and you get a definition.

\begin{Shaded}
\begin{Highlighting}[]
\NormalTok{mpg}
\end{Highlighting}
\end{Shaded}

\begin{verbatim}
## # A tibble: 234 x 11
##    manufacturer model      displ  year   cyl trans drv     cty   hwy fl    class
##    <chr>        <chr>      <dbl> <int> <int> <chr> <chr> <int> <int> <chr> <chr>
##  1 audi         a4           1.8  1999     4 auto~ f        18    29 p     comp~
##  2 audi         a4           1.8  1999     4 manu~ f        21    29 p     comp~
##  3 audi         a4           2    2008     4 manu~ f        20    31 p     comp~
##  4 audi         a4           2    2008     4 auto~ f        21    30 p     comp~
##  5 audi         a4           2.8  1999     6 auto~ f        16    26 p     comp~
##  6 audi         a4           2.8  1999     6 manu~ f        18    26 p     comp~
##  7 audi         a4           3.1  2008     6 auto~ f        18    27 p     comp~
##  8 audi         a4 quattro   1.8  1999     4 manu~ 4        18    26 p     comp~
##  9 audi         a4 quattro   1.8  1999     4 auto~ 4        16    25 p     comp~
## 10 audi         a4 quattro   2    2008     4 manu~ 4        20    28 p     comp~
## # ... with 224 more rows
\end{verbatim}

\begin{Shaded}
\begin{Highlighting}[]
\FunctionTok{library}\NormalTok{(tidyverse)}
\FunctionTok{library}\NormalTok{(ggplot2)}

\NormalTok{carData }\OtherTok{\textless{}{-}} \FunctionTok{ggplot}\NormalTok{(}\AttributeTok{data =}\NormalTok{ mpg, }\AttributeTok{mapping =} \FunctionTok{aes}\NormalTok{(}\AttributeTok{x =}\NormalTok{ displ, }\AttributeTok{y =}\NormalTok{ hwy)) }\SpecialCharTok{+}
  \FunctionTok{geom\_point}\NormalTok{(}\AttributeTok{mapping =} \FunctionTok{aes}\NormalTok{(}\AttributeTok{color =}\NormalTok{ class, }\AttributeTok{shape =}\NormalTok{ class)) }\SpecialCharTok{+}
  \FunctionTok{geom\_smooth}\NormalTok{(}\AttributeTok{data =} \FunctionTok{filter}\NormalTok{(mpg, class }\SpecialCharTok{==} \StringTok{"subcompact"}\NormalTok{), }\AttributeTok{se =} \ConstantTok{FALSE}\NormalTok{)}

\NormalTok{carData}
\end{Highlighting}
\end{Shaded}

\begin{verbatim}
## `geom_smooth()` using method = 'loess' and formula 'y ~ x'
\end{verbatim}

\begin{verbatim}
## Warning: The shape palette can deal with a maximum of 6 discrete values because
## more than 6 becomes difficult to discriminate; you have 7. Consider
## specifying shapes manually if you must have them.
\end{verbatim}

\begin{verbatim}
## Warning: Removed 62 rows containing missing values (geom_point).
\end{verbatim}

\includegraphics{notesFirstWeek_files/figure-latex/carData-1.pdf}

This carData graph uses data mpg, which also has a mapping argument,
this makes the data ``Mapped'' to certain properties. In this case, the
x and y axis is displ and hwy, this is easily exchanged for other shit.
geom point is the dots on the plot, which can be any shape / color
depending on how you set it. You can either go LGBTQ+ or set it to a
random variable. Up to you. Geom Smooth is the line given, which we can
use the filter function to find and make a line of best fit for the shit
that we want. In this case, we have to declare the data used and what
category that we want.

idk what the fuck se is lmao

\hypertarget{facuet-failure.}{%
\subsection{Facuet Failure.}\label{facuet-failure.}}

Facet is a way to seperate the data into different graphs so its not
clunky as fuck. Facet\_wrap is to set how many rows int he data. We can
also put same x and y variables in the data to save space.

\includegraphics{notesFirstWeek_files/figure-latex/pressure-1.pdf}

\begin{Shaded}
\begin{Highlighting}[]
\FunctionTok{ggplot}\NormalTok{(}\AttributeTok{data =}\NormalTok{ mpg) }\SpecialCharTok{+} 
  \FunctionTok{geom\_point}\NormalTok{(}\AttributeTok{mapping =} \FunctionTok{aes}\NormalTok{(}\AttributeTok{x =}\NormalTok{ displ, }\AttributeTok{y =}\NormalTok{ hwy)) }\SpecialCharTok{+} 
  \FunctionTok{facet\_grid}\NormalTok{(drv }\SpecialCharTok{\textasciitilde{}}\NormalTok{ cyl)}
\end{Highlighting}
\end{Shaded}

\includegraphics{notesFirstWeek_files/figure-latex/unnamed-chunk-2-1.pdf}

What does facet wrap do? fuck use this ? but, it wraps your data into
different sets. Making the shit look pretty.

\hypertarget{stuff-from-the-lecture.}{%
\subsection{Stuff from the lecture.}\label{stuff-from-the-lecture.}}

Using the data given from thomas, we are given an encoder and a data of
the pulses of some people.

\begin{Shaded}
\begin{Highlighting}[]
\NormalTok{dat }\OtherTok{=} \FunctionTok{read.table}\NormalTok{(}\StringTok{"pulse.dat"}\NormalTok{, }\AttributeTok{header=}\ConstantTok{TRUE}\NormalTok{)}
\end{Highlighting}
\end{Shaded}

Something that reads the data, a header is a logical variable which
shows which working directory its in.

\begin{Shaded}
\begin{Highlighting}[]
\FunctionTok{attach}\NormalTok{(dat)}
\end{Highlighting}
\end{Shaded}

We are making the data set saved, and easy to find.

\begin{Shaded}
\begin{Highlighting}[]
\FunctionTok{stem}\NormalTok{(dat}\SpecialCharTok{$}\NormalTok{pulse)}
\end{Highlighting}
\end{Shaded}

\begin{verbatim}
## 
##   The decimal point is 1 digit(s) to the right of the |
## 
##   5 | 56
##   6 | 12
##   6 | 55689
##   7 | 133
##   7 | 667777
##   8 | 01112
##   8 | 556667
##   9 | 3
\end{verbatim}

Creates a stem and leaf plot, from the sub category pulse, pulse has
been ``Head'' From the data, which means we take a single column from
the data set.

\begin{Shaded}
\begin{Highlighting}[]
\FunctionTok{hist}\NormalTok{(dat}\SpecialCharTok{$}\NormalTok{pulse, }\AttributeTok{main =} \StringTok{"Student Pulse rates (bpm)"}\NormalTok{)}
\end{Highlighting}
\end{Shaded}

\includegraphics{notesFirstWeek_files/figure-latex/histogram-1.pdf}
Creates a histogram of the data, which in this case is Pulse

\begin{Shaded}
\begin{Highlighting}[]
\FunctionTok{boxplot}\NormalTok{(dat}\SpecialCharTok{$}\NormalTok{pulse, }\AttributeTok{horizontal =} \ConstantTok{TRUE}\NormalTok{,}
        \AttributeTok{main =} \StringTok{"Student Pulse rates (bpm)"}\NormalTok{)}
\end{Highlighting}
\end{Shaded}

\includegraphics{notesFirstWeek_files/figure-latex/boxPlot-1.pdf} Same
shit, but a box plot.

\begin{Shaded}
\begin{Highlighting}[]
\FunctionTok{summary}\NormalTok{(dat}\SpecialCharTok{$}\NormalTok{pulse)}
\end{Highlighting}
\end{Shaded}

\begin{verbatim}
##    Min. 1st Qu.  Median    Mean 3rd Qu.    Max. 
##   55.00   68.25   77.00   75.23   81.75   93.00
\end{verbatim}

Gives General Information of the data set, which we can see from the
results the Min, 1st Quartile, Median, Mean, 3rd Quartile and Max is
showing

\begin{Shaded}
\begin{Highlighting}[]
\FunctionTok{sd}\NormalTok{(dat}\SpecialCharTok{$}\NormalTok{pulse)}
\end{Highlighting}
\end{Shaded}

\begin{verbatim}
## [1] 9.729739
\end{verbatim}

Standard deviation.

\begin{Shaded}
\begin{Highlighting}[]
\FunctionTok{mean}\NormalTok{(dat}\SpecialCharTok{$}\NormalTok{pulse)}
\end{Highlighting}
\end{Shaded}

\begin{verbatim}
## [1] 75.23333
\end{verbatim}

Mean

This is Thomas' T test, I have altered it slightly to automate to reject
or do not reject the null hypothesis.

\begin{Shaded}
\begin{Highlighting}[]
\NormalTok{dat1 }\OtherTok{=} \FunctionTok{read.table}\NormalTok{(}\StringTok{"encoder.txt"}\NormalTok{, }\AttributeTok{header =} \ConstantTok{TRUE}\NormalTok{)}
\FunctionTok{str}\NormalTok{(dat1) }\CommentTok{\# Look at the data object}
\end{Highlighting}
\end{Shaded}

\begin{verbatim}
## 'data.frame':    10 obs. of  1 variable:
##  $ time: num  18.3 17.9 19.1 16.8 18.9 17.4 19.6 18.3 19.6 16.3
\end{verbatim}

\begin{Shaded}
\begin{Highlighting}[]
\NormalTok{dat1 }\CommentTok{\# Look at all observations}
\end{Highlighting}
\end{Shaded}

\begin{verbatim}
##    time
## 1  18.3
## 2  17.9
## 3  19.1
## 4  16.8
## 5  18.9
## 6  17.4
## 7  19.6
## 8  18.3
## 9  19.6
## 10 16.3
\end{verbatim}

\begin{Shaded}
\begin{Highlighting}[]
\FunctionTok{head}\NormalTok{(dat1) }\CommentTok{\# display first six observations}
\end{Highlighting}
\end{Shaded}

\begin{verbatim}
##   time
## 1 18.3
## 2 17.9
## 3 19.1
## 4 16.8
## 5 18.9
## 6 17.4
\end{verbatim}

\begin{Shaded}
\begin{Highlighting}[]
\FunctionTok{summary}\NormalTok{(dat1}\SpecialCharTok{$}\NormalTok{time) }\CommentTok{\# obtain descriptive statistics}
\end{Highlighting}
\end{Shaded}

\begin{verbatim}
##    Min. 1st Qu.  Median    Mean 3rd Qu.    Max. 
##   16.30   17.52   18.30   18.22   19.05   19.60
\end{verbatim}

\begin{Shaded}
\begin{Highlighting}[]
\FunctionTok{sd}\NormalTok{(dat1}\SpecialCharTok{$}\NormalTok{time)}\SpecialCharTok{\^{}}\DecValTok{2} \CommentTok{\#Find the SD.}
\end{Highlighting}
\end{Shaded}

\begin{verbatim}
## [1] 1.281778
\end{verbatim}

\begin{Shaded}
\begin{Highlighting}[]
\NormalTok{pValue }\OtherTok{\textless{}{-}} \FunctionTok{t.test}\NormalTok{(dat1}\SpecialCharTok{$}\NormalTok{time, }\AttributeTok{mu =} \FloatTok{19.25}\NormalTok{)}\SpecialCharTok{$}\NormalTok{p.value}
\CommentTok{\# ifelse(bob \textless{} 0.05 \textless{}{-}yes = print("Reject"), no = print("Do not reject"))}
\ControlFlowTok{if}\NormalTok{(pValue }\SpecialCharTok{\textless{}} \FloatTok{0.05}\NormalTok{) \{}
  \FunctionTok{print}\NormalTok{(}\StringTok{"Reject"}\NormalTok{)}
\NormalTok{\} }\ControlFlowTok{else}\NormalTok{ \{}
  \FunctionTok{print}\NormalTok{(}\StringTok{"Do Not Reject"}\NormalTok{)}
\NormalTok{\}}
\end{Highlighting}
\end{Shaded}

\begin{verbatim}
## [1] "Reject"
\end{verbatim}

\begin{Shaded}
\begin{Highlighting}[]
\CommentTok{\# t.test(dat1$time, mu = 19.25, alternative = "less")}

\CommentTok{\# t.test(dat1$time, mu = 19.25, alternative = "greater")}
\end{Highlighting}
\end{Shaded}

Okay, we didn't know why the fuck mu is 19.25, mu is the mean of the
POPULATION When we find the descriptive statistics of the data, we
realize the mean is 18.22, lol what the fuck right? No, this is coz this
is this the mean of the SAMPLE

When we are constructing our hypothesis, we will need to create a Null
Hypothesis and a Alternative Hypothsis.

In this case, the sample mean is NOT 19.25, but 18.22 from our t test.
Therefore, We reject the null hypothesis AND the sample mean was
significantly less than the population

Anyway, thats all the notes for this week. Thomas, fuck you.

\end{document}
